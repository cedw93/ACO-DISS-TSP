\chapter{Testing PUT THIS IN OWN FILE}

Detailed descriptions of every test case are definitely not what is required here. What is important is to show that you adopted a sensible strategy that was, in principle, capable of testing the system adequately even if you did not have the time to test the system fully.

Have you tested your system on �real users�? For example, if your system is supposed to solve a problem for a business, then it would be appropriate to present your approach to involve the users in the testing process and to record the results that you obtained. Depending on the level of detail, it is likely that you would put any detailed results in an appendix.

The following sections indicate some areas you might include. Other sections may be more appropriate to your project. 

\section{Overall Approach to Testing}

\section{Automated Testing}

\subsection{Unit Tests}

\subsection{User Interface Testing}

\subsection{Stress Testing}

\subsection{Other types of testing}

\section{Integration Testing}

\section{User Testing}

\chapter{Evaluation}

Examiners expect to find in your dissertation a section addressing such questions as:

\begin{itemize}
   \item Were the requirements correctly identified? 
   \item Were the design decisions correct?
   \item Could a more suitable set of tools have been chosen?
   \item How well did the software meet the needs of those who were expecting to use it?
   \item How well were any other project aims achieved?
   \item If you were starting again, what would you do differently?
\end{itemize}

Such material is regarded as an important part of the dissertation; it should demonstrate that you are capable not only of carrying out a piece of work but also of thinking critically about how you did it and how you might have done it better. This is seen as an important part of an honours degree. 

There will be good things and room for improvement with any project. As you write this section, identify and discuss the parts of the work that went well and also consider ways in which the work could be improved. 

Review the discussion on the Evaluation section from the lectures. A recording is available on Blackboard. 