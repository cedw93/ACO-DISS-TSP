\chapter{Testing}
\renewcommand{\thechapter}{\Alph{chapter}}

\section{Test Code Examples}

\begin{figure}[H]
\begin{lstlisting}
@Test
public void testValidationRulesAllowLegalAlphaValues(){
	double[] values = new double[]{2.5, -2.1, 4.999999999, -4.999999999, 4, -2, -1.65, 5, -5, 3, -3.21};
	for(int i = 0; i < values.length; i++){
		assertTrue("validaite should return true alpha is legal", aco.validate(values[i], 2.5, 0.2, 0.2, 10, 15, 3, 3));
	}
}
\end{lstlisting}
\caption{Code representing the automated testing process for checking mutliple alpha parameter values.}
\label{testAlpha}
\end{figure}

\begin{figure}[H]
\begin{lstlisting}
@Test
public void testValidationRulesAllowLegalBetaValues(){
	double[] values = new double[]{2.5, -2.1, 4.999999999, -4.999999999, 4, -2, -1.65, 5, -5, 3, -3.21};
	for(int i = 0; i < values.length; i++){
		assertTrue("validaite should return true Beta is legal", aco.validate(2, values[i], 0.2, 0.2, 10, 15, 3, 3));
	}
}
\end{lstlisting}
\caption{Code representing the automated testing process for checking mutliple beta parameter values.}
\label{testBeta}
\end{figure}

\begin{figure}[H]
\begin{lstlisting}
@Test
public void testValidationRulesAllowLegalAgentsValues(){
	int[] values = new int[]{25, 13, 19, 48, 29, 23, 11, 8, 2, 33, 40, 39, 6};
	for(int i = 0; i < values.length; i++){
		assertTrue("validaite should return true number of agents is legal", aco.validate(2, 2.5, 0.2, 0.2, values[i], 15, 3, 3));
	}
}

@Test
public void testValidationRulesShouldNotAllowIllegalAgentsValues(){
	int[] values = new int[]{-1, -12, 55, 100, 87, 69, 52, 99, -72, -12, -5, 58, 60};
	for(int i = 0; i < values.length; i++){
		assertFalse("validaite should return true number of agents is illegal", aco.validate(2, 2.5, 0.2, 0.2, values[i], 15, 3, 3));
	}
}
\end{lstlisting}
\caption{Code representing the automated testing process for checking mutliple agent parameter values.}
\label{testAgent}
\end{figure}

\begin{figure}[H]
\begin{lstlisting}
@Test
public void testTotalProbabiliyIsCalculatedToEqualOne(){
	double value = 1.1275;
	double result = ((Math.pow(value, world.getAlpha())) * (Math.pow(value, world.getBeta())));
	assertEquals(1.82221,result, 0.0001);

	value = 3.1275;

	double result2 = ((Math.pow(value, world.getAlpha())) * (Math.pow(value, world.getBeta())));
	assertEquals(299.2172,result2, 0.0001);

	value = 2.1983;

	double result3 = ((Math.pow(value, world.getAlpha())) * (Math.pow(value, world.getBeta())));
	assertEquals(51.337509,result3, 0.0001);

	value = 0.0213;
	//test one with lower values 10^-9 is the result
	double result4 = ((Math.pow(value, world.getAlpha())) * (Math.pow(value, world.getBeta())));
	assertEquals(0.00000000438427,result4, 0.000000001);

	double totalSum = result + result2 + result3 + result4;

	assertEquals(352.376919, totalSum, 0.0001);

	double p1 = result/totalSum;
	double p2 = result2/totalSum;
	double p3 = result3/totalSum;
	double p4 = result4/totalSum;

	//check that total probability sums up to 1.0
	assertEquals(1.0, p1 + p2 + p3 + p4, 0.0001);
}
\end{lstlisting}
\caption{Code representing the automated testing process for checking mutliple agent parameter values.}
\label{testProb}
\end{figure}

\section{Black-box Testing}
\subsection{File Content Examples}
\label{fileIOtest}

The below figure \ref{validConfig} contains an example of a valid problem configuration. Each line of content refers to a different specific value. A correct and valid file should contain values in this order and format;

\begin{itemize}
\item double, the alpha parameter value between must be in the range of $-5.0\ to\ 5.0$
\item double, the double parameter value between must be in the range of $-5.0\ to\ 5.0$
\item double, the decay rate value must be in the range of $0\ to\ 1$
\item double, the initial edge pheromone value must be in the range of $0\ to\ 1$
\item int, the number of agents must be between $1\ and\ 50$
\item int, the number of cities must be between $3\ and\ 25$
\item int. int, the X and Y coordinate for a City. There should be 1 line for the corresponding number of cities defined in the previous line
\item int, the number of uphill routes to be generated this must be between $0\ and\ 15$
\item int, the number of iterations this must be greater than $0$
\item EOF, signifies that this is the end of the configuration content
\end{itemize}

\begin{figure}[H]
0.2 \\
2.0 \\
0.2 \\
0.8 \\
30 \\
cities \\
15 \\
36 6 \\
30 1 \\
19 1 \\
25 22 \\
8 7 \\
2 20 \\
26 19 \\
19 3 \\
24 23 \\
7 9 \\
12 6 \\
18 27 \\
2 24 \\
35 23 \\
21 27 \\
7 \\
5 \\
EOF
\caption{Contents of a legal problem representation stored in a file.}
\label{validConfig}
\end{figure}

Below is an example of one of the test files used to ensure that only legal configurations are accepted. This file is rejected because the number of agents is specified as 30sffs which is not a valid integer value. It is easier to reject this and tell the user to correct it than it is to extract the integer value from a malformed integer.

\begin{figure}[H]
0.2 \\
2.0 \\
0.2 \\
0.8 \\
30sffs \\
cities \\
15 \\
36 6 \\
30 1 \\
19 1 \\
25 22 \\
8 7 \\
2 20 \\
26 19 \\
19 3 \\
24 23 \\
7 9 \\
12 6 \\
18 27 \\
2 24 \\
35 23 \\
21 27 \\
7 \\
5 \\
EOF
\caption{Contents of an illegal problem representation stored in a file. The number of agents in line 5 is not an integer value (30sffs)}
\label{invalidConfigagentsNFE}
\end{figure}

Below is another example of an illegal configuration. The contents of this file contains no value representing the number of agents.

\begin{figure}[H]
0.2 \\
2.0 \\
0.2 \\
0.8 \\
cities \\
15 \\
36 6 \\
30 1 \\
19 1 \\
25 22 \\
8 7 \\
2 20 \\
26 19 \\
19 3 \\
24 23 \\
7 9 \\
12 6 \\
18 27 \\
2 24 \\
35 23 \\
21 27 \\
7 \\
5 \\
EOF
\caption{Contents of an illegal problem representation stored in a file. The number of agents has not been defined.}
\label{invalidConfigMissing}
\end{figure}

\subsection{Acceptance Tests}
\label{AcceptanceTestz}

\begin{table}[h]
\centering
\begin{tabular}{|l|l|l|l|}
\hline
\textbf{Requirement} & \multicolumn{1}{c|}{\textbf{Test}}                                                                          & \multicolumn{1}{c|}{\textbf{Pass/Fail}} & \multicolumn{1}{c|}{\textbf{Comment}}                                                                                                                                                                                                                          \\ \hline
          \textbf{FR1}   & \begin{tabular}[c]{@{}l@{}}Test that the application\\ can be launches from the\\ .jar file\end{tabular}    & Pass                                    & \begin{tabular}[c]{@{}l@{}}This will be the default\\ way to launch the \\ application.\end{tabular} \\ \hline
\textbf{FR16}        & \begin{tabular}[c]{@{}l@{}}The user is able to exit\\ the application whenever the\\ wish to\end{tabular}          & Pass                                    & \begin{tabular}[c]{@{}l@{}}Click the exit on the main\\ JFrame container. This will\\ exit the application regardless\\ of what the application is \\ currently doing.\end{tabular} \\ \hline
                                                                                                                                                    
\end{tabular}
\caption[FR1 and FR16  acceptance tests]{A summary of the tests used to acceptance test against FR1 and FR16}
\end{table}


\begin{table}[h]
\centering
\begin{tabular}{|l|l|l|l|}
\hline
\textbf{Requirement} & \multicolumn{1}{c|}{\textbf{Test}}                                                                          & \multicolumn{1}{c|}{\textbf{Pass/Fail}} & \multicolumn{1}{c|}{\textbf{Comment}}                                                                                                                                                                                                                          \\ \hline
\textbf{FR2} & \begin{tabular}[c]{@{}l@{}}A World can be\\ randomly generated\end{tabular}                                 & Pass                                    & \begin{tabular}[c]{@{}l@{}}Clicking 'start'\\ will cause a random world\\ to be generated.\end{tabular}                                                                                                                                                       \\ \hline
\end{tabular}
\caption[FR2 acceptance tests]{A summary of the tests used to acceptance test against FR2}
\end{table}


\begin{table}[H]
\centering
\begin{tabular}{|l|l|l|l|}
\hline
\textbf{Requirement} & \multicolumn{1}{c|}{\textbf{Test}}                                                                          & \multicolumn{1}{c|}{\textbf{Pass/Fail}} & \multicolumn{1}{c|}{\textbf{Comment}}                                                                                                                                                                                                                               \\ \hline
\textbf{FR3, FR5}         & \begin{tabular}[c]{@{}l@{}}World visualisation: Cities are\\ displayed in the correct location\end{tabular} & Pass                                    & \begin{tabular}[c]{@{}l@{}}The correct number of \\ cities is displayed to the \\ user and each city is shown\\ at its own location. \\ Occasionally two cities \\ may render close to each other.\end{tabular}                                                     \\ \hline
\textbf{FR3, FR5}         & \begin{tabular}[c]{@{}l@{}}World visualisation: The paths\\ between city locations are shown\end{tabular}   & Pass                                    & \begin{tabular}[c]{@{}l@{}}The graph is a fully connected \\ graph so a path is drawn from a \\ city to every other city.This is not \\ always clear as some paths may \\ overlap.\end{tabular}                                                                     \\ \hline
\textbf{FR3, FR5}         & \begin{tabular}[c]{@{}l@{}}World visualisation: The correct\\ number of agents is shown\end{tabular}        & Pass                                    & \begin{tabular}[c]{@{}l@{}}The correct number of agents\\ is displayed to the user, As \\ multiple agents can be shown \\ at the same city index, there\\ is a text based count which \\ shows that the correct number \\ of agents has been rendered.\end{tabular} \\ \hline
\end{tabular}
\caption[FR3 and FR5 acceptance tests]{A summary of the tests used to acceptance test against FR3 and FR5. These requirements are closely related thus they can be tested in parallel}
\end{table}

\begin{table}[H]
\centering
\begin{tabular}{|l|l|l|l|}
\hline
\textbf{Requirement} & \multicolumn{1}{c|}{\textbf{Test}}                                                                                           & \multicolumn{1}{c|}{\textbf{Pass/Fail}} & \multicolumn{1}{c|}{\textbf{Comment}}                                           \\ \hline
\textbf{FR4}         & \begin{tabular}[c]{@{}l@{}}Ensure that the user\\ can define a value for \\ the 'alpha' parameter\end{tabular}               & Pass                                    & \begin{tabular}[c]{@{}l@{}}This is complete\\ with error checking\end{tabular}  \\ \hline
\textbf{FR4}         & \begin{tabular}[c]{@{}l@{}}Ensure that the user\\ can define a value for\\  the 'beta' parameter\end{tabular}                & Pass                                    & \begin{tabular}[c]{@{}l@{}}This is complete\\ with error checking\end{tabular}  \\ \hline
\textbf{FR4}         & \begin{tabular}[c]{@{}l@{}}Ensure that the user\\ can define a value for \\ the 'decay rate'\\  parameter\end{tabular}       & Pass                                    & \begin{tabular}[c]{@{}l@{}}This is complete\\ with error checking\end{tabular}  \\ \hline
\textbf{FR4}         & \begin{tabular}[c]{@{}l@{}}Ensure that the user\\ can define a value for \\ the 'initial pheromone'\\ parameter\end{tabular} & Pass                                    & \begin{tabular}[c]{@{}l@{}}This is complete\\ with error checking\end{tabular}  \\ \hline
\textbf{FR4}         & \begin{tabular}[c]{@{}l@{}}Ensure that the user\\ can define the number\\ of 'uphill' paths\end{tabular}                     & Pass                                    & \begin{tabular}[c]{@{}l@{}}This is complete \\ with error checking\end{tabular} \\ \hline
\textbf{FR4}         & \begin{tabular}[c]{@{}l@{}}Ensure the user can\\ define the number of\\ agents\end{tabular}                                  & Pass                                    & \begin{tabular}[c]{@{}l@{}}This is complete\\ with error checking\end{tabular}  \\ \hline
\textbf{FR4}         & \begin{tabular}[c]{@{}l@{}}Ensure the user can\\ define the number of\\ cities\end{tabular}                                  & Pass                                    & \begin{tabular}[c]{@{}l@{}}This is complete\\ with error checking\end{tabular}  \\ \hline
\textbf{FR4}         & \begin{tabular}[c]{@{}l@{}}Ensure the user can\\ specify the number \\ of elite agents\end{tabular}                          & Pass                                    & \begin{tabular}[c]{@{}l@{}}This is complete\\ with error checking\end{tabular}  \\ \hline
\end{tabular}
\caption[FR4 acceptance tests]{A summary of the tests used to acceptance test against FR4}
\end{table}

\begin{table}[H]
\centering
\begin{tabular}{|l|l|l|l|}
\hline
\textbf{Requirement} & \multicolumn{1}{c|}{\textbf{Test}}                                                                               & \multicolumn{1}{c|}{\textbf{Pass/Fail}} & \multicolumn{1}{c|}{\textbf{Comment}}                                                                                               \\ \hline
\textbf{FR6, FR13}         & \begin{tabular}[c]{@{}l@{}}The opacity of the\\ path is relative to the\\ pheromone on each \\ edge\end{tabular} & Pass                                    & \begin{tabular}[c]{@{}l@{}}Initially this will\\ be the value specified\\ by the 'initialPhero' \\ parameter\end{tabular}           \\ \hline
\textbf{FR6, FR13}         & \begin{tabular}[c]{@{}l@{}}Pheromone decay is\\ correctly visualised\end{tabular}                                & Pass                                    & \begin{tabular}[c]{@{}l@{}}As the algorithm \\ executed there is a clear\\ representation of the\\ pheromone decaying.\end{tabular} \\ \hline
\end{tabular}
\caption[FR6 and FR13 acceptance tests]{A summary of the tests used to acceptance test against FR6 and FR13}
\end{table}

\begin{table}[H]
\centering
\begin{tabular}{|l|l|l|l|}
\hline
\textbf{Requirement} & \multicolumn{1}{c|}{\textbf{Test}}                                                                      & \multicolumn{1}{c|}{\textbf{Pass/Fail}} & \multicolumn{1}{c|}{\textbf{Comment}}                                                                                \\ \hline
\textbf{FR7, FR12}         & \begin{tabular}[c]{@{}l@{}}The agents can be displayed\\ at any city location\end{tabular}              & Pass                                    & \begin{tabular}[c]{@{}l@{}}Regardless of how many\\ agents are at a city, only 1\\ image will be shown.\end{tabular} \\ \hline
\textbf{FR7, FR12}         & \begin{tabular}[c]{@{}l@{}}The path the agent took\\ between cities will be \\ highlighted\end{tabular} & Pass                                    &                                                                                                                      \\ \hline
\end{tabular}
\caption[FR7 and FR12 acceptance tests]{A summary of the tests used to acceptance test against FR7 and FR12}
\end{table}

\begin{table}[H]
\centering
\begin{tabular}{|l|l|l|l|}
\hline
\textbf{Requirement} & \multicolumn{1}{c|}{\textbf{Test}}                                                                & \multicolumn{1}{c|}{\textbf{Pass/Fail}} & \multicolumn{1}{c|}{\textbf{Comment}}                                                                                      \\ \hline
\textbf{FR8}         & \begin{tabular}[c]{@{}l@{}}Ensure the user can\\ start the algorithms \\ exectution\end{tabular}  & Pass                                    & \begin{tabular}[c]{@{}l@{}}Clicking the \\ 'start' button will \\ signal the algorithms \\ execution to begin\end{tabular} \\ \hline
\textbf{FR9}         & \begin{tabular}[c]{@{}l@{}}Ensure the user can\\ stop an already running\\ algorithm\end{tabular} & Pass                                    & \begin{tabular}[c]{@{}l@{}}An algorithm can \\ only be stopped if there\\ is one already running.\end{tabular}             \\ \hline
\end{tabular}
\caption[FR8 and FR9 acceptance tests]{A summary of the tests used to acceptance test against FR8 and FR9}
\end{table}

\begin{table}[H]
\centering
\begin{tabular}{|l|l|l|l|}
\hline
\textbf{Requirement} & \multicolumn{1}{c|}{\textbf{Test}}                                                                         & \multicolumn{1}{c|}{\textbf{Pass/Fail}} & \multicolumn{1}{c|}{\textbf{Comment}}                                                 \\ \hline
\textbf{FR10}        & \begin{tabular}[c]{@{}l@{}}Ensure the algorithm\\ rejects an invalid alpha\\ value\end{tabular}            & Pass                                    & \begin{tabular}[c]{@{}l@{}}An error message\\ is displayed to the\\ user\end{tabular} \\ \hline
\textbf{FR10}        & \begin{tabular}[c]{@{}l@{}}Ensure the algorithm\\ rejects an invalid beta\\ value\end{tabular}             & Pass                                    & \begin{tabular}[c]{@{}l@{}}An error message\\ is displayed to the\\ user\end{tabular} \\ \hline
\textbf{FR10}        & \begin{tabular}[c]{@{}l@{}}Ensure the algorithm\\ rejects an invalid decay\\ rate\end{tabular}             & Pass                                    & \begin{tabular}[c]{@{}l@{}}An error message\\ is displayed to the\\ user\end{tabular} \\ \hline
\textbf{FR10}        & \begin{tabular}[c]{@{}l@{}}Ensure the algorithm\\ rejects an invalid number\\ of iterations\end{tabular}   & Pass                                    & \begin{tabular}[c]{@{}l@{}}An error message\\ is displayed to the\\ user\end{tabular} \\ \hline
\textbf{FR10}        & \begin{tabular}[c]{@{}l@{}}Ensure the algorithm\\ rejects an invalid number\\ of uphill paths\end{tabular} & Pass                                    & \begin{tabular}[c]{@{}l@{}}An error message\\ is displayed to the\\ user\end{tabular} \\ \hline
\textbf{FR10}        & \begin{tabular}[c]{@{}l@{}}Ensure the algorithm\\ rejects an invalid number\\ of cities\end{tabular}       & Pass                                    & \begin{tabular}[c]{@{}l@{}}An error message\\ is displayed to the\\ user\end{tabular} \\ \hline
\textbf{FR10}        & \begin{tabular}[c]{@{}l@{}}Ensure the algorithm\\ rejects an invalid number\\ of agents\end{tabular}       & Pass                                    & \begin{tabular}[c]{@{}l@{}}An error message\\ is displayed to the\\ user\end{tabular} \\ \hline
\end{tabular}
\caption[FR10 acceptance tests]{A summary of the tests used to acceptance test against FR10}
\end{table}

\begin{table}[H]
\centering
\begin{tabular}{|l|l|l|l|}
\hline
\textbf{Requirement} & \multicolumn{1}{c|}{\textbf{Test}}                                                                                        & \multicolumn{1}{c|}{\textbf{Pass/Fail}} & \multicolumn{1}{c|}{\textbf{Comment}}                                                                                                       \\ \hline
\textbf{FR11}        & \begin{tabular}[c]{@{}l@{}}The best route is displayed\\ to the user as a sequence of \\ city indexes\end{tabular}        & Pass                                    & \begin{tabular}[c]{@{}l@{}}This is displayed in\\ addition to the visual \\ route\end{tabular}                                              \\ \hline
\textbf{FR11}        & \begin{tabular}[c]{@{}l@{}}The best route is shown as\\ a red line between the cities\\ along the best route\end{tabular} & Pass                                    & \begin{tabular}[c]{@{}l@{}}This line is thicker than \\ the lines used to represent\\ the regular paths so it is easy\\ to see\end{tabular} \\ \hline
\end{tabular}
\caption[FR11 acceptance tests]{A summary of the tests used to acceptance test against FR11}
\end{table}

\begin{table}[H]
\centering
\begin{tabular}{|l|l|l|l|}
\hline
\textbf{Requirement} & \multicolumn{1}{c|}{\textbf{Test}}                                                                       & \multicolumn{1}{c|}{\textbf{Pass/Fail}} & \multicolumn{1}{c|}{\textbf{Comment}}                                                                                                                                               \\ \hline
\textbf{FR14}        & \begin{tabular}[c]{@{}l@{}}A problem configuration\\ can be loaded from an \\ external file\end{tabular} & Pass                                    & \begin{tabular}[c]{@{}l@{}}This test was performed \\ with the knowledge that \\ the files contents would enable\\ the loading to complete as all \\ values were legal\end{tabular} \\ \hline
\textbf{FR15}        & \begin{tabular}[c]{@{}l@{}}The current problem can be\\ written to a file\end{tabular}                   & Pass                                    & \begin{tabular}[c]{@{}l@{}}This is only possible if a problem\\ has been generated.\end{tabular}                                                                                    \\ \hline
\end{tabular}
\caption[FR14 and FR15 acceptance tests]{A summary of the tests used to acceptance test against FR14 and FR15}
\end{table}

\subsection{Interface Testing}
\label{UITESTSM8}
