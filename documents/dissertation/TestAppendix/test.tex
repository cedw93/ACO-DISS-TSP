\chapter{Testing}
\renewcommand{\thechapter}{\Alph{chapter}}

\section{Test Code Examples}

\begin{figure}[H]
\begin{lstlisting}
@Test
public void testValidationRulesAllowLegalAlphaValues(){
	double[] values = new double[]{2.5, -2.1, 4.999999999, -4.999999999, 4, -2, -1.65, 5, -5, 3, -3.21};
	for(int i = 0; i < values.length; i++){
		assertTrue("validaite should return true alpha is legal", aco.validate(values[i], 2.5, 0.2, 0.2, 10, 15, 3, 3));
	}
}
\end{lstlisting}
\caption{Code representing the automated testing process for checking mutliple alpha parameter values.}
\label{testAlpha}
\end{figure}

\begin{figure}[H]
\begin{lstlisting}
@Test
public void testValidationRulesAllowLegalBetaValues(){
	double[] values = new double[]{2.5, -2.1, 4.999999999, -4.999999999, 4, -2, -1.65, 5, -5, 3, -3.21};
	for(int i = 0; i < values.length; i++){
		assertTrue("validaite should return true Beta is legal", aco.validate(2, values[i], 0.2, 0.2, 10, 15, 3, 3));
	}
}
\end{lstlisting}
\caption{Code representing the automated testing process for checking mutliple beta parameter values.}
\label{testBeta}
\end{figure}

\begin{figure}[H]
\begin{lstlisting}
@Test
public void testValidationRulesAllowLegalAgentsValues(){
	int[] values = new int[]{25, 13, 19, 48, 29, 23, 11, 8, 2, 33, 40, 39, 6};
	for(int i = 0; i < values.length; i++){
		assertTrue("validaite should return true number of agents is legal", aco.validate(2, 2.5, 0.2, 0.2, values[i], 15, 3, 3));
	}
}

@Test
public void testValidationRulesShouldNotAllowIllegalAgentsValues(){
	int[] values = new int[]{-1, -12, 55, 100, 87, 69, 52, 99, -72, -12, -5, 58, 60};
	for(int i = 0; i < values.length; i++){
		assertFalse("validaite should return true number of agents is illegal", aco.validate(2, 2.5, 0.2, 0.2, values[i], 15, 3, 3));
	}
}
\end{lstlisting}
\caption{Code representing the automated testing process for checking mutliple agent parameter values.}
\label{testAgent}
\end{figure}

\begin{figure}[H]
\begin{lstlisting}
@Test
public void testTotalProbabiliyIsCalculatedToEqualOne(){
	double value = 1.1275;
	double result = ((Math.pow(value, world.getAlpha())) * (Math.pow(value, world.getBeta())));
	assertEquals(1.82221,result, 0.0001);

	value = 3.1275;

	double result2 = ((Math.pow(value, world.getAlpha())) * (Math.pow(value, world.getBeta())));
	assertEquals(299.2172,result2, 0.0001);

	value = 2.1983;

	double result3 = ((Math.pow(value, world.getAlpha())) * (Math.pow(value, world.getBeta())));
	assertEquals(51.337509,result3, 0.0001);

	value = 0.0213;
	//test one with lower values 10^-9 is the result
	double result4 = ((Math.pow(value, world.getAlpha())) * (Math.pow(value, world.getBeta())));
	assertEquals(0.00000000438427,result4, 0.000000001);

	double totalSum = result + result2 + result3 + result4;

	assertEquals(352.376919, totalSum, 0.0001);

	double p1 = result/totalSum;
	double p2 = result2/totalSum;
	double p3 = result3/totalSum;
	double p4 = result4/totalSum;

	//check that total probability sums up to 1.0
	assertEquals(1.0, p1 + p2 + p3 + p4, 0.0001);
}
\end{lstlisting}
\caption{Code representing the automated testing process for checking mutliple agent parameter values.}
\label{testProb}
\end{figure}

\section{Black-box Testing}
\subsection{File Content Examples}
\label{fileIOtest}

The below figure \ref{validConfig} contains an example of a valid problem configuration. Each line of content refers to a different specific value. A correct and valid file should contain values in this order and format;

\begin{itemize}
\item double, the alpha parameter value between must be in the range of $-5.0\ to\ 5.0$
\item double, the double parameter value between must be in the range of $-5.0\ to\ 5.0$
\item double, the decay rate value must be in the range of $0\ to\ 1$
\item double, the initial edge pheromone value must be in the range of $0\ to\ 1$
\item int, the number of agents must be between $1\ and\ 50$
\item int, the number of cities must be between $3\ and\ 25$
\item int. int, the X and Y coordinate for a City. There should be 1 line for the corresponding number of cities defined in the previous line
\item int, the number of uphill routes to be generated this must be between $0\ and\ 15$
\item int, the number of iterations this must be greater than $0$
\item EOF, signifies that this is the end of the configuration content
\end{itemize}

\begin{figure}[H]
0.2 \\
2.0 \\
0.2 \\
0.8 \\
30 \\
cities \\
15 \\
36 6 \\
30 1 \\
19 1 \\
25 22 \\
8 7 \\
2 20 \\
26 19 \\
19 3 \\
24 23 \\
7 9 \\
12 6 \\
18 27 \\
2 24 \\
35 23 \\
21 27 \\
7 \\
5 \\
EOF
\caption{Contents of a legal problem representation stored in a file.}
\label{validConfig}
\end{figure}

Below is an example of one of the test files used to ensure that only legal configurations are accepted. This file is rejected because the number of agents is specified as 30sffs which is not a valid integer value. It is easier to reject this and tell the user to correct it than it is to extract the integer value from a malformed integer.

\begin{figure}[H]
0.2 \\
2.0 \\
0.2 \\
0.8 \\
30sffs \\
cities \\
15 \\
36 6 \\
30 1 \\
19 1 \\
25 22 \\
8 7 \\
2 20 \\
26 19 \\
19 3 \\
24 23 \\
7 9 \\
12 6 \\
18 27 \\
2 24 \\
35 23 \\
21 27 \\
7 \\
5 \\
EOF
\caption{Contents of an illegal problem representation stored in a file. The number of agents in line 5 is not an integer value (30sffs)}
\label{invalidConfigagentsNFE}
\end{figure}

Below is another example of an illegal configuration. The contents of this file contains no value representing the number of agents.

\begin{figure}[H]
0.2 \\
2.0 \\
0.2 \\
0.8 \\
cities \\
15 \\
36 6 \\
30 1 \\
19 1 \\
25 22 \\
8 7 \\
2 20 \\
26 19 \\
19 3 \\
24 23 \\
7 9 \\
12 6 \\
18 27 \\
2 24 \\
35 23 \\
21 27 \\
7 \\
5 \\
EOF
\caption{Contents of an illegal problem representation stored in a file. The number of agents has not been defined.}
\label{invalidConfigMissing}
\end{figure}

\subsection{Acceptance Tests}
\label{AcceptanceTestz}

\begin{table}[h]
\centering
\begin{tabular}{|l|l|l|l|}
\hline
\textbf{Requirement} & \multicolumn{1}{c|}{\textbf{Test}}                                                                          & \multicolumn{1}{c|}{\textbf{Pass/Fail}} & \multicolumn{1}{c|}{\textbf{Comment}}                                                                                                                                                                                                                          \\ \hline
          \textbf{FR1}   & \begin{tabular}[c]{@{}l@{}}Test that the application\\ can be launches from the\\ .jar file\end{tabular}    & Pass                                    & \begin{tabular}[c]{@{}l@{}}This will be the default\\ way to launch the \\ application.\end{tabular} \\ \hline
\textbf{FR16}        & \begin{tabular}[c]{@{}l@{}}The user is able to exit\\ the application whenever the\\ wish to\end{tabular}          & Pass                                    & \begin{tabular}[c]{@{}l@{}}Click the exit on the main\\ JFrame container. This will\\ exit the application regardless\\ of what the application is \\ currently doing.\end{tabular} \\ \hline
                                                                                                                                                    
\end{tabular}
\caption[FR1 and FR16  acceptance tests]{A summary of the tests used to acceptance test against FR1 and FR16}
\end{table}


\begin{table}[h]
\centering
\begin{tabular}{|l|l|l|l|}
\hline
\textbf{Requirement} & \multicolumn{1}{c|}{\textbf{Test}}                                                                          & \multicolumn{1}{c|}{\textbf{Pass/Fail}} & \multicolumn{1}{c|}{\textbf{Comment}}                                                                                                                                                                                                                          \\ \hline
\textbf{FR2} & \begin{tabular}[c]{@{}l@{}}A World can be\\ randomly generated\end{tabular}                                 & Pass                                    & \begin{tabular}[c]{@{}l@{}}Clicking 'start'\\ will cause a random world\\ to be generated.\end{tabular}                                                                                                                                                       \\ \hline
\end{tabular}
\caption[FR2 acceptance tests]{A summary of the tests used to acceptance test against FR2}
\end{table}


\begin{table}[H]
\centering
\begin{tabular}{|l|l|l|l|}
\hline
\textbf{Requirement} & \multicolumn{1}{c|}{\textbf{Test}}                                                                          & \multicolumn{1}{c|}{\textbf{Pass/Fail}} & \multicolumn{1}{c|}{\textbf{Comment}}                                                                                                                                                                                                                               \\ \hline
\textbf{FR3, FR5}         & \begin{tabular}[c]{@{}l@{}}World visualisation: Cities are\\ displayed in the correct location\end{tabular} & Pass                                    & \begin{tabular}[c]{@{}l@{}}The correct number of \\ cities is displayed to the \\ user and each city is shown\\ at its own location. \\ Occasionally two cities \\ may render close to each other.\end{tabular}                                                     \\ \hline
\textbf{FR3, FR5}         & \begin{tabular}[c]{@{}l@{}}World visualisation: The paths\\ between city locations are shown\end{tabular}   & Pass                                    & \begin{tabular}[c]{@{}l@{}}The graph is a fully connected \\ graph so a path is drawn from a \\ city to every other city.This is not \\ always clear as some paths may \\ overlap.\end{tabular}                                                                     \\ \hline
\textbf{FR3, FR5}         & \begin{tabular}[c]{@{}l@{}}World visualisation: The correct\\ number of agents is shown\end{tabular}        & Pass                                    & \begin{tabular}[c]{@{}l@{}}The correct number of agents\\ is displayed to the user, As \\ multiple agents can be shown \\ at the same city index, there\\ is a text based count which \\ shows that the correct number \\ of agents has been rendered.\end{tabular} \\ \hline
\end{tabular}
\caption[FR3 and FR5 acceptance tests]{A summary of the tests used to acceptance test against FR3 and FR5. These requirements are closely related thus they can be tested in parallel}
\end{table}

\begin{table}[H]
\centering
\begin{tabular}{|l|l|l|l|}
\hline
\textbf{Requirement} & \multicolumn{1}{c|}{\textbf{Test}}                                                                                           & \multicolumn{1}{c|}{\textbf{Pass/Fail}} & \multicolumn{1}{c|}{\textbf{Comment}}                                           \\ \hline
\textbf{FR4}         & \begin{tabular}[c]{@{}l@{}}Ensure that the user\\ can define a value for \\ the 'alpha' parameter\end{tabular}               & Pass                                    & \begin{tabular}[c]{@{}l@{}}This is complete\\ with error checking\end{tabular}  \\ \hline
\textbf{FR4}         & \begin{tabular}[c]{@{}l@{}}Ensure that the user\\ can define a value for\\  the 'beta' parameter\end{tabular}                & Pass                                    & \begin{tabular}[c]{@{}l@{}}This is complete\\ with error checking\end{tabular}  \\ \hline
\textbf{FR4}         & \begin{tabular}[c]{@{}l@{}}Ensure that the user\\ can define a value for \\ the 'decay rate'\\  parameter\end{tabular}       & Pass                                    & \begin{tabular}[c]{@{}l@{}}This is complete\\ with error checking\end{tabular}  \\ \hline
\textbf{FR4}         & \begin{tabular}[c]{@{}l@{}}Ensure that the user\\ can define a value for \\ the 'initial pheromone'\\ parameter\end{tabular} & Pass                                    & \begin{tabular}[c]{@{}l@{}}This is complete\\ with error checking\end{tabular}  \\ \hline
\textbf{FR4}         & \begin{tabular}[c]{@{}l@{}}Ensure that the user\\ can define the number\\ of 'uphill' paths\end{tabular}                     & Pass                                    & \begin{tabular}[c]{@{}l@{}}This is complete \\ with error checking\end{tabular} \\ \hline
\textbf{FR4}         & \begin{tabular}[c]{@{}l@{}}Ensure the user can\\ define the number of\\ agents\end{tabular}                                  & Pass                                    & \begin{tabular}[c]{@{}l@{}}This is complete\\ with error checking\end{tabular}  \\ \hline
\textbf{FR4}         & \begin{tabular}[c]{@{}l@{}}Ensure the user can\\ define the number of\\ cities\end{tabular}                                  & Pass                                    & \begin{tabular}[c]{@{}l@{}}This is complete\\ with error checking\end{tabular}  \\ \hline
\textbf{FR4}         & \begin{tabular}[c]{@{}l@{}}Ensure the user can\\ specify the number \\ of elite agents\end{tabular}                          & Pass                                    & \begin{tabular}[c]{@{}l@{}}This is complete\\ with error checking\end{tabular}  \\ \hline
\end{tabular}
\caption[FR4 acceptance tests]{A summary of the tests used to acceptance test against FR4}
\end{table}

\begin{table}[H]
\centering
\begin{tabular}{|l|l|l|l|}
\hline
\textbf{Requirement} & \multicolumn{1}{c|}{\textbf{Test}}                                                                               & \multicolumn{1}{c|}{\textbf{Pass/Fail}} & \multicolumn{1}{c|}{\textbf{Comment}}                                                                                               \\ \hline
\textbf{FR6, FR13}         & \begin{tabular}[c]{@{}l@{}}The opacity of the\\ path is relative to the\\ pheromone on each \\ edge\end{tabular} & Pass                                    & \begin{tabular}[c]{@{}l@{}}Initially this will\\ be the value specified\\ by the 'initialPhero' \\ parameter\end{tabular}           \\ \hline
\textbf{FR6, FR13}         & \begin{tabular}[c]{@{}l@{}}Pheromone decay is\\ correctly visualised\end{tabular}                                & Pass                                    & \begin{tabular}[c]{@{}l@{}}As the algorithm \\ executed there is a clear\\ representation of the\\ pheromone decaying.\end{tabular} \\ \hline
\end{tabular}
\caption[FR6 and FR13 acceptance tests]{A summary of the tests used to acceptance test against FR6 and FR13}
\end{table}

\begin{table}[H]
\centering
\begin{tabular}{|l|l|l|l|}
\hline
\textbf{Requirement} & \multicolumn{1}{c|}{\textbf{Test}}                                                                      & \multicolumn{1}{c|}{\textbf{Pass/Fail}} & \multicolumn{1}{c|}{\textbf{Comment}}                                                                                \\ \hline
\textbf{FR7, FR12}         & \begin{tabular}[c]{@{}l@{}}The agents can be displayed\\ at any city location\end{tabular}              & Pass                                    & \begin{tabular}[c]{@{}l@{}}Regardless of how many\\ agents are at a city, only 1\\ image will be shown.\end{tabular} \\ \hline
\textbf{FR7, FR12}         & \begin{tabular}[c]{@{}l@{}}The path the agent took\\ between cities will be \\ highlighted\end{tabular} & Pass                                    &                                                                                                                      \\ \hline
\end{tabular}
\caption[FR7 and FR12 acceptance tests]{A summary of the tests used to acceptance test against FR7 and FR12}
\end{table}

\begin{table}[H]
\centering
\begin{tabular}{|l|l|l|l|}
\hline
\textbf{Requirement} & \multicolumn{1}{c|}{\textbf{Test}}                                                                & \multicolumn{1}{c|}{\textbf{Pass/Fail}} & \multicolumn{1}{c|}{\textbf{Comment}}                                                                                      \\ \hline
\textbf{FR8}         & \begin{tabular}[c]{@{}l@{}}Ensure the user can\\ start the algorithms \\ exectution\end{tabular}  & Pass                                    & \begin{tabular}[c]{@{}l@{}}Clicking the \\ 'start' button will \\ signal the algorithms \\ execution to begin\end{tabular} \\ \hline
\textbf{FR9}         & \begin{tabular}[c]{@{}l@{}}Ensure the user can\\ stop an already running\\ algorithm\end{tabular} & Pass                                    & \begin{tabular}[c]{@{}l@{}}An algorithm can \\ only be stopped if there\\ is one already running.\end{tabular}             \\ \hline
\end{tabular}
\caption[FR8 and FR9 acceptance tests]{A summary of the tests used to acceptance test against FR8 and FR9}
\end{table}

\begin{table}[H]
\centering
\begin{tabular}{|l|l|l|l|}
\hline
\textbf{Requirement} & \multicolumn{1}{c|}{\textbf{Test}}                                                                         & \multicolumn{1}{c|}{\textbf{Pass/Fail}} & \multicolumn{1}{c|}{\textbf{Comment}}                                                 \\ \hline
\textbf{FR10}        & \begin{tabular}[c]{@{}l@{}}Ensure the algorithm\\ rejects an invalid alpha\\ value\end{tabular}            & Pass                                    & \begin{tabular}[c]{@{}l@{}}An error message\\ is displayed to the\\ user\end{tabular} \\ \hline
\textbf{FR10}        & \begin{tabular}[c]{@{}l@{}}Ensure the algorithm\\ rejects an invalid beta\\ value\end{tabular}             & Pass                                    & \begin{tabular}[c]{@{}l@{}}An error message\\ is displayed to the\\ user\end{tabular} \\ \hline
\textbf{FR10}        & \begin{tabular}[c]{@{}l@{}}Ensure the algorithm\\ rejects an invalid decay\\ rate\end{tabular}             & Pass                                    & \begin{tabular}[c]{@{}l@{}}An error message\\ is displayed to the\\ user\end{tabular} \\ \hline
\textbf{FR10}        & \begin{tabular}[c]{@{}l@{}}Ensure the algorithm\\ rejects an invalid number\\ of iterations\end{tabular}   & Pass                                    & \begin{tabular}[c]{@{}l@{}}An error message\\ is displayed to the\\ user\end{tabular} \\ \hline
\textbf{FR10}        & \begin{tabular}[c]{@{}l@{}}Ensure the algorithm\\ rejects an invalid number\\ of uphill paths\end{tabular} & Pass                                    & \begin{tabular}[c]{@{}l@{}}An error message\\ is displayed to the\\ user\end{tabular} \\ \hline
\textbf{FR10}        & \begin{tabular}[c]{@{}l@{}}Ensure the algorithm\\ rejects an invalid number\\ of cities\end{tabular}       & Pass                                    & \begin{tabular}[c]{@{}l@{}}An error message\\ is displayed to the\\ user\end{tabular} \\ \hline
\textbf{FR10}        & \begin{tabular}[c]{@{}l@{}}Ensure the algorithm\\ rejects an invalid number\\ of agents\end{tabular}       & Pass                                    & \begin{tabular}[c]{@{}l@{}}An error message\\ is displayed to the\\ user\end{tabular} \\ \hline
\end{tabular}
\caption[FR10 acceptance tests]{A summary of the tests used to acceptance test against FR10}
\end{table}

\begin{table}[H]
\centering
\begin{tabular}{|l|l|l|l|}
\hline
\textbf{Requirement} & \multicolumn{1}{c|}{\textbf{Test}}                                                                                        & \multicolumn{1}{c|}{\textbf{Pass/Fail}} & \multicolumn{1}{c|}{\textbf{Comment}}                                                                                                       \\ \hline
\textbf{FR11}        & \begin{tabular}[c]{@{}l@{}}The best route is displayed\\ to the user as a sequence of \\ city indexes\end{tabular}        & Pass                                    & \begin{tabular}[c]{@{}l@{}}This is displayed in\\ addition to the visual \\ route\end{tabular}                                              \\ \hline
\textbf{FR11}        & \begin{tabular}[c]{@{}l@{}}The best route is shown as\\ a red line between the cities\\ along the best route\end{tabular} & Pass                                    & \begin{tabular}[c]{@{}l@{}}This line is thicker than \\ the lines used to represent\\ the regular paths so it is easy\\ to see\end{tabular} \\ \hline
\end{tabular}
\caption[FR11 acceptance tests]{A summary of the tests used to acceptance test against FR11}
\end{table}

\begin{table}[H]
\centering
\begin{tabular}{|l|l|l|l|}
\hline
\textbf{Requirement} & \multicolumn{1}{c|}{\textbf{Test}}                                                                       & \multicolumn{1}{c|}{\textbf{Pass/Fail}} & \multicolumn{1}{c|}{\textbf{Comment}}                                                                                                                                               \\ \hline
\textbf{FR14}        & \begin{tabular}[c]{@{}l@{}}A problem configuration\\ can be loaded from an \\ external file\end{tabular} & Pass                                    & \begin{tabular}[c]{@{}l@{}}This test was performed \\ with the knowledge that \\ the files contents would enable\\ the loading to complete as all \\ values were legal\end{tabular} \\ \hline
\textbf{FR15}        & \begin{tabular}[c]{@{}l@{}}The current problem can be\\ written to a file\end{tabular}                   & Pass                                    & \begin{tabular}[c]{@{}l@{}}This is only possible if a problem\\ has been generated.\end{tabular}                                                                                    \\ \hline
\end{tabular}
\caption[FR14 and FR15 acceptance tests]{A summary of the tests used to acceptance test against FR14 and FR15}
\end{table}

\subsection{Interface Testing}
\label{UITESTSM8}

The following tests are not directly related to the functional requirements however, they provide a huge increase in the overall qualtiy of the application as proper error feedback is essential. The parameters and file IO have both been tested for boundary conditions and the like in the unit tests for the application and as a result the author is not concerned about the values tested below. The purpose of these tests is to solely ensure the view is performiing as expected.

\begin{table}[H]
\centering
\begin{tabular}{|l|l|l|l|}
\hline
\multicolumn{4}{|l|}{\textbf{Parameter Error feedback (1)}}                                                                                                                                                                                                                                                                                                                                                                     \\ \hline
\textbf{Test}                                                                                          & \textbf{Expected Result}                                                                                                             & \textbf{Pass/fail} & \textbf{Comments}                                                                                                                                      \\ \hline
\begin{tabular}[c]{@{}l@{}}Input an alpha value \\ that is too hgih\end{tabular}                       & \begin{tabular}[c]{@{}l@{}}An error prompt should\\ display informing the \\ user of an illegal alpha\\ value\end{tabular}           & Pass               &                                                                                                                                                        \\ \hline
\begin{tabular}[c]{@{}l@{}}Input an alpha value\\ that is too low\end{tabular}                         & \begin{tabular}[c]{@{}l@{}}An error prompt should\\ display informing the \\ user of an illegal alpha\\ value\end{tabular}           & Pass               &                                                                                                                                                        \\ \hline
\begin{tabular}[c]{@{}l@{}}Input a non-double\\ for the alpha value\end{tabular}                       & \begin{tabular}[c]{@{}l@{}}An error prompt should\\ display informing the \\ user of an invalid data\\ type\end{tabular}             & Pass               &                                                                                                                                                        \\ \hline
\begin{tabular}[c]{@{}l@{}}Input a beta value\\ that is too high\end{tabular}                          & \begin{tabular}[c]{@{}l@{}}An error prompt should\\ display informing the \\ user of an illegal beta\\ value\end{tabular}            & Pass               &                                                                                                                                                        \\ \hline
\begin{tabular}[c]{@{}l@{}}Input a beta value\\ that is too low\end{tabular}                           & \begin{tabular}[c]{@{}l@{}}An error prompt should\\ display informing the \\ user of an illegal beta\\ value\end{tabular}            & Pass               &                                                                                                                                                        \\ \hline
\begin{tabular}[c]{@{}l@{}}Input a non-double\\ for the beta value\end{tabular}                        & \begin{tabular}[c]{@{}l@{}}An error prompt should\\ display informing the\\ user of an invalid data\\ type\end{tabular}              &Pass                  &                                                                                                                                                        \\ \hline
\begin{tabular}[c]{@{}l@{}}Specify less than\\ zero iterations\end{tabular}                            & \begin{tabular}[c]{@{}l@{}}An error prompt should\\ display informing the \\ user of an illegal \\ number of iterations\end{tabular} & Pass               &                                                                                                                                                        \\ \hline
\begin{tabular}[c]{@{}l@{}}Input a non integer\\ value for the number\\ of iterations\end{tabular}     & \begin{tabular}[c]{@{}l@{}}An error prompt should\\ display informing the\\ user of an invalid \\ data type\end{tabular}             & Pass               &                                                                                                                                                        \\ \hline
\begin{tabular}[c]{@{}l@{}}Input a decay rate\\ that is too low\end{tabular}                           & \begin{tabular}[c]{@{}l@{}}An error prompt should\\ display informing the\\ user of an illegal \\ decay rate value\end{tabular}      & Pass               &                                                                                                                                                        \\ \hline
\begin{tabular}[c]{@{}l@{}}Input a decay rate\\ that is too high\end{tabular}                          & \begin{tabular}[c]{@{}l@{}}An error prompt should\\ display informing the\\ user of an illegal\\ decay rate value\end{tabular}       & Pass               &                                                                                                                                                        \\ \hline
\begin{tabular}[c]{@{}l@{}}Input a non-double \\ value for decay rate\end{tabular}                     & \begin{tabular}[c]{@{}l@{}}An error prompt should\\ display informing the\\ user of an invalid data\\ type\end{tabular}              & Pass               &                                                                                                                                                        \\ \hline
\end{tabular}
\caption[Error feedback testing (1)]{A summary of the tests used to see if the error feedback is as expected(1).}
\end{table}

\begin{table}[H]
\centering
\begin{tabular}{|l|l|l|l|}
\hline
\multicolumn{4}{|l|}{\textbf{Parameter Error feedback (2)}}                                                                                                                                                                                                                                                                                                                                                                     \\ \hline
\textbf{Test}                                                                                          & \textbf{Expected Result}                                                                                                             & \textbf{Pass/fail} & \textbf{Comments}                                                                                                                                      \\ \hline
\begin{tabular}[c]{@{}l@{}}Specify less than \\ zero agents\end{tabular}                               & \begin{tabular}[c]{@{}l@{}}An error prompt should\\ display informing the \\ user of an illegal \\ number of agents\end{tabular}     & Fail               & \begin{tabular}[c]{@{}l@{}}The prompt displayed\\ when the agent value is \\ illegal was incorrectly\\ formatted.\\ This has been fixed\end{tabular}   \\ \hline
\begin{tabular}[c]{@{}l@{}}Specify too many\\ agents\end{tabular}                                      & \begin{tabular}[c]{@{}l@{}}An error prompt should\\ display informing the \\ user of an illegal \\ number of agents\end{tabular}     & Fail               & \begin{tabular}[c]{@{}l@{}}The prompt displayed\\ when the agent value \\ is illegal was incorrectly\\ formatted.  \\ This has been fixed\end{tabular} \\ \hline
\begin{tabular}[c]{@{}l@{}}Specify a non-integer\\ value for the number\\ of agents\end{tabular}       & \begin{tabular}[c]{@{}l@{}}An error prompt should\\ display informing the\\ user of an invalid data\\ type\end{tabular}              & Pass               &                                                                                                                                                        \\ \hline
Specify too few cities                                                                                 & \begin{tabular}[c]{@{}l@{}}An error prompt should\\ display informing the\\ user of an illegal \\ number of cities\end{tabular}      & Pass               &                                                                                                                                                        \\ \hline
\begin{tabular}[c]{@{}l@{}}Specify too many\\ cities\end{tabular}                                      & \begin{tabular}[c]{@{}l@{}}An error prompt should\\ display informing the\\ user of an illegal\\ number of cities\end{tabular}       & Pass               &                                                                                                                                                        \\ \hline
\begin{tabular}[c]{@{}l@{}}Specify a non-integer\\ value for the number\\ of cities\end{tabular}       & \begin{tabular}[c]{@{}l@{}}An error prompt should\\ display informing the\\ user of an invalid \\ data type\end{tabular}             & Pass               &                                                                                                                                                        \\ \hline
\begin{tabular}[c]{@{}l@{}}Specify too few uphill\\ paths\end{tabular}                                 & \begin{tabular}[c]{@{}l@{}}An error prompt should\\ display informing the\\ user of an illegal\\ number of uphill paths\end{tabular} & Pass               &                                                                                                                                                        \\ \hline
\begin{tabular}[c]{@{}l@{}}Specify too many uphill\\ paths\end{tabular}                                & \begin{tabular}[c]{@{}l@{}}An error prompt should\\ display informing the\\ user of an illegal\\ number of uphill paths\end{tabular} & Pass               &                                                                                                                                                        \\ \hline
\begin{tabular}[c]{@{}l@{}}Specify a non-integer\\ value for the number of\\ uphill paths\end{tabular} & \begin{tabular}[c]{@{}l@{}}An error prompt should\\ display informing the\\ user of an invalid\\ data type\end{tabular}              & Pass               &                                                                                                                                                        \\ \hline
\end{tabular}
\caption[Error feedback testing (2)]{A summary of the tests used to see if the error feedback is as expected(2).}
\end{table}

\begin{table}[H]
\centering
\begin{tabular}{|l|l|l|l|}
\hline
\multicolumn{4}{|l|}{\textbf{File IO Error Feedback}}                                                                                                                                                                                                                                                                                                                                                                                                         \\ \hline
\textbf{Test}                                                                                                & \textbf{Expected Result}                                                                                                                                & \textbf{Pass/fail} & \textbf{Comments}                                                                                                                                               \\ \hline
\begin{tabular}[c]{@{}l@{}}Attempt to load \\ whilst the algorithm\\  is running\end{tabular}                & \begin{tabular}[c]{@{}l@{}}An error message\\ should tell the user\\ that the algorithm must\\ be stopped or complete\\ before loading\end{tabular}     & Pass               &                                                                                                                                                                 \\ \hline
\begin{tabular}[c]{@{}l@{}}Attempt to load \\ an invalid file\end{tabular}                                   & \begin{tabular}[c]{@{}l@{}}An error message should\\ be displayed informing\\ the user than the problem\\ configuration is invalid\end{tabular}         & Pass               & \begin{tabular}[c]{@{}l@{}}The cause of the error\\ is unimportant, checks have\\ been done already in regards\\ to catching illegal file content.\end{tabular} \\ \hline
\begin{tabular}[c]{@{}l@{}}Attempt to save\\ whilst the algorithm\\ is running\end{tabular}                  & \begin{tabular}[c]{@{}l@{}}An error message\\ should tell the user\\ that the algorithm must\\ be stopped or complete\\ before saving\end{tabular}      & Pass               &                                                                                                                                                                 \\ \hline
\begin{tabular}[c]{@{}l@{}}Attempt to save\\ when there is nothing \\ to save (no world exists)\end{tabular} & \begin{tabular}[c]{@{}l@{}}The user should be informed\\ that there is nothing to save\\ and should be instructed\\ how to create a world.\end{tabular} & Pass               &                                                                                                                                                                 \\ \hline
\begin{tabular}[c]{@{}l@{}}A loaded problem should\\ be rendered\end{tabular}                                & \begin{tabular}[c]{@{}l@{}}If the loading of a \\ configuration completes,\\ the configuration should\\ be displayed to the user\end{tabular}           & Pass               & \begin{tabular}[c]{@{}l@{}}The parameter text fields\\ are also updated to reflect the\\ values contained in such file\end{tabular}                             \\ \hline
\end{tabular}
\caption[FileIO Error Testing]{A summary of the tests used to see if the error feedback is as expected.}
\end{table}

The tests below are designed to see if the algorithm handled general rendering tasks as expected. This does not includes the rendering of the problem or the movement of agents or the like as this have been tested during the acceptance testing phase.

\begin{table}[H]

\begin{tabular}{|l|l|l|l|}
\hline
\multicolumn{4}{|l|}{\textbf{General View Testing(1)}}                                                                                                                                                                                                                                                                                                                                                                                                                                           \\ \hline
\textbf{Test}                                                                                                            & \textbf{Expected Result}                                                                                                                                                                                                   & \textbf{Pass/fail} & \textbf{Comments}                                                                                                \\ \hline
\begin{tabular}[c]{@{}l@{}}When step mode is\\ enabled the view\\ changes states\end{tabular}                            & \begin{tabular}[c]{@{}l@{}}When step mode is enabled\\ the 'start' button should now\\ read 'step'\end{tabular}                                                                                                            & Pass               &                                                                                                                  \\ \hline
\begin{tabular}[c]{@{}l@{}}When step mode is\\ disabled the view \\ changes state\end{tabular}                           & \begin{tabular}[c]{@{}l@{}}When step mode is disabled\\ the 'step' button should read\\ 'start'\end{tabular}                                                                                                               & Pass               &                                                                                                                  \\ \hline
\begin{tabular}[c]{@{}l@{}}The text values are\\ correctly reset\end{tabular}                                            & \begin{tabular}[c]{@{}l@{}}When the user presses the\\ 'reset values' button, the text\\ fields are restored to deafult\end{tabular}                                                                                       & Fail               & \begin{tabular}[c]{@{}l@{}}The iterations field was\\ not being reset. This has\\ since been fixed.\end{tabular} \\ \hline
\begin{tabular}[c]{@{}l@{}}When in step mode\\ the user cant modify the\\ aglorithm\end{tabular}                         & \begin{tabular}[c]{@{}l@{}}When the user is in step\\ mode and has started the \\ iteration process the text \\ fields should become disabled\\ until the algorithm \\ completes or the user presses\\  stop.\end{tabular} & Pass               &                                                                                                                  \\ \hline
\begin{tabular}[c]{@{}l@{}}When in step mode is \\ complete the view \\ should update\end{tabular}                       & \begin{tabular}[c]{@{}l@{}}When step mode is complete,\\ the text fields should become \\ usable again\end{tabular}                                                                                                        & Pass               &                                                                                                                  \\ \hline
\begin{tabular}[c]{@{}l@{}}When in step mode the\\ algorithm should execute\\ accordingly\end{tabular}                   & \begin{tabular}[c]{@{}l@{}}When the user has enabled \\ step mode, the algorithm should\\ solve on a step-by-step \\ basis and should not \\ automatically execute until\\ completion.\end{tabular}                        & Pass               & \begin{tabular}[c]{@{}l@{}}The user can stop this\\ at any time.\end{tabular}                                    \\ \hline
\begin{tabular}[c]{@{}l@{}}The additional information\\ should correctly update as the\\ aglorithm executed\end{tabular} & \begin{tabular}[c]{@{}l@{}}As the algorithm executes these\\ values should change from their \\ defaults to correctly respent the \\ current state of the aglorithm\end{tabular}                                           & Pass               &                                                                                                                  \\ \hline
\begin{tabular}[c]{@{}l@{}}The speed of execution can\\ be changed freely\end{tabular}                                   & \begin{tabular}[c]{@{}l@{}}The user should be able to\\ dynamically change the speed\\ of which the algorithm executes\end{tabular}                                                                                        & Pass               &                                                                                                                  \\ \hline
\end{tabular}
\caption[General View Testing(1)]{A summary of the tests used to see if view renders elements as expected(1).}
\end{table}

\begin{table}[H]

\begin{tabular}{|l|l|l|l|}
\hline
\multicolumn{4}{|l|}{\textbf{General View Testing(2)}}                                                                                                                                                                                                                                                                                                                                                                                                                                           \\ \hline
\textbf{Test}                                                                                                            & \textbf{Expected Result}                                                                                                                                                                                                   & \textbf{Pass/fail} & \textbf{Comments}                                                                                                \\ \hline
\begin{tabular}[c]{@{}l@{}}When uphill paths are disabled\\ the view should update\end{tabular}                          & \begin{tabular}[c]{@{}l@{}}When the user has disabled \\ uphill path generation, the text\\ field responsible for setting the \\ number of uphill paths should\\ be disabled\end{tabular}                                  & Pass               &                                                                                                                  \\ \hline
\begin{tabular}[c]{@{}l@{}}When uphill paths are enabled\\ the view should update\end{tabular}                           & \begin{tabular}[c]{@{}l@{}}When the user has re-enabled \\ the uphill path generation, the \\ text field responsible for setting \\ the number of uphill paths \\ should no longer be disabled\end{tabular}                & Pass               &                                                                                                                  \\ \hline
\end{tabular}
\caption[General View Testing(2)]{A summary of the tests used to see if view renders elements as expected(2).}
\end{table}

