\chapter{Implemented Code}
\renewcommand{\thechapter}{\Alph{chapter}}

\begin{figure}[H]
\begin{lstlisting}
pheromone = new Pheromone[cities.size()][cities.size()];
	for(int i = 0; i < pheromone.length; i++){
		for(int j = 0; j < pheromone[0].length; j++){
			pheromone[i][j] = new Pheromone(aco.getInitialPheromone());
		}
	}
\end{lstlisting}
\caption{Snippet of code used to inialise the pheromone Matrix. $cities$ is a list of $City$ obejcts. $aco$ is the instance of the $AntColonyOptimisation$ Class}
\label{initPheroCode}
\end{figure}

\begin{figure}[H]
\begin{lstlisting}
distanceMatrix = new double[cities.size()][cities.size()];
	for(int i = 0; i < distanceMatrix.length; i++){
		for(int j = 0; j < distanceMatrix[0].length; j++){
			distanceMatrix[i][j] = Globals.calculateEuclidianDistance(cities.get(i).getX(), cities.get(i).getY(), cities.get(j).getX(), cities.get(j).getY());
		}
	}
\end{lstlisting}
\caption{Snippet of code used to inialise the distance Matrix. $cities$ is a list of $City$ obejcts.}
\label{initDistanceCode}
\end{figure}

\begin{figure}[H]
\begin{lstlisting}
invertedMatrix = new double[distanceMatrix.length][distanceMatrix[0].length];	
	for(int i = 0; i < distanceMatrix.length; i++){
		for(int j = 0; j < distanceMatrix[0].length; j++){
			invertedMatrix[i][j] = invertValue(distanceMatrix[i][j]);
		}
	}
\end{lstlisting}
\caption{Snippet of code used to inialise the inverted distance Matrix. $cities$ is a list of $City$ obejcts. $invertValue{distanceMatrix[i][j]}$ returns $\frac{1}{distanceMatrix[i][j]}$}
\label{initInverteddistanceCode}
\end{figure}

\begin{figure}[H]
\begin{lstlisting}
cities = new ArrayList<City>();
	Random r = new Random();
	for(int i = 0; i < numberOfCities; i++){
		// the (+1) is to stop cities having the index '0' which would cause them to half render out of view
		int x = r.nextInt(aco.getBoundaryX()) + 1;
		int y = r.nextInt(aco.getBoundaryY()) + 1;
		//make sure 2 cities can't spawn on top of each other
		for(City city: cities){
			if(x == city.getX() && y == city.getY()){
				x = r.nextInt(aco.getBoundaryX()) + 1;
				y = r.nextInt(aco.getBoundaryY()) + 1;
			}
		}
		cities.add(new City(x,y,i));	
	}
\end{lstlisting}
\caption{Snippet of code used to inialise the collection of $City$ obejcts.}
\label{initCity}
\end{figure}

\begin{figure}[H]
\begin{lstlisting}
Random r = new Random();
	ants = new ArrayList<Ant>(numberOfAnts);
	for(int i = 0; i < numberOfAnts; i++){
		int index = r.nextInt(cities.size());
		for(City c: cities){
			if(index == c.getIndex()){
				c.adjustAntsHere(1);
				break;
			}
		}
		ants.add(new Ant(this, index));
	}
\end{lstlisting}
\caption{Snippet of code used to inialise the collection of $Ant$ obejcts. $this$ refers to the current $World$ Object.}
\label{initAnt}
\end{figure}

\begin{figure}[H]
\begin{lstlisting}
for(int i = 0; i < pheromone.length; i++){
		for(int j = 0; j < pheromone[0].length; j++){
			updatePheromone(i, j, pheromone[i][j].getNewPhero());
			//reset the new phero values
			pheromone[i][j].resetNewPhero();
		}
	}

public void updatePheromone(int x, int y, double newPheromone) {
	double phero = calculatePheromones(pheromone[x][y].getPheromoneValue(), newPheromone);
	//if phero is not negative then update the current concentration
	//if phero is negative then just set it as 0, you can't have negative phero on an edge
	if (phero >= 0.0d) {
		pheromone[x][y].setPheromoneValue(phero);
	} else {
		pheromone[x][y].setPheromoneValue(0.0d);
	}
}

public double calculatePheromones(double current, double newPheromone) {
	//we dont need to store the result in a temporary variable, just return the equation in place
	return ((1 - aco.getDecayRate()) * current + newPheromone);
}	
\end{lstlisting}
\caption{Snippet of code used to update the pheromone values for every edge. $aco$ is the current instance of the $AntColonyOptimisation$ Class.}
\label{codephero}
\end{figure}

\begin{figure}[H]
\begin{lstlisting}
public final int getNextProbableNode(int y) {
	//This is an adapted version of a similar method provided by Thomas Jungblut found here: https://code.google.com/p/antcolonyopt/
	//create a location to store the probability for all next locations
	//this can then be easily accessed to return the next index for the ant's move
	//only do this if there is in fact locations to move to
	if (unvisited > 0) {
		int danglingUnvisited = -1;
		final double[] weights = new double[visited.length];
		double columnSum = 0.0d;
		for (int i = 0; i < visited.length; i++) {
			columnSum += calculateIndividualProbability(y, i);
		}
		//once we have the value for sum (which the sum off all solutions evaluation)
		double sum = 0.0d;
		for (int x = 0; x < visited.length; x++) {
			if (!visited[x]) {
				weights[x] = calculateTotalProbability(x, y, columnSum);
				sum += weights[x];
				danglingUnvisited = x;
			}
		}
		//if sum is 0 then return, as this will be used in a division it cannot be zero
		if (sum == 0.0d)
			return danglingUnvisited;
		/*  We need to give each index of the probability weighting based on the result of calculateToalProbability
		 *  this result is then divided by the total sum of all probabilities (usually 1)
		 *  The result of this division is then put in the correct index in the matrix in order to get the probability weighting
		 *
		 */
		double pSum = 0.0d;
		for (int i = 0; i < visited.length; i++) {
			pSum += weights[i] / sum;
			weights[i] = pSum;
		}
		//nextDouble returns a value between 0 and 1, so this can be used to effectively select a 'random' index based on the weighted probability
		//provided virtually as is by Thomas Jungblut: https://code.google.com/p/antcolonyopt/
		//this is what makes the ant walk 'randomly'
		final double r = random.nextDouble();
		for (int i = 0; i < visited.length; i++) {
			if (!visited[i]) {
				if (r <= weights[i]) {
					return i;
				}
			}
		}
	}
	return -1;
}
\end{lstlisting}
\caption{Snippet of code used to select the Ants next location. This is a modifiedf version of the code provided by Thomas Jungblut \cite{tjung:aco:blog}}
\label{nextNodeCode}
\end{figure}

\begin{figure}[H]
\begin{lstlisting}
boolean reset = false;
for(int i = 0; i < iterations; i++){
		aco.setCurrentIteration(i);
		if(aco.getRunning()){
			ArrayList<Ant> ants = (ArrayList<Ant>)aco.getWorld().getAnts();
			antsWorking = aco.getNoOfAgents();
			if(reset){
				//for the next iteration re-init the ants and go again
				for(City c: aco.getWorld().getCities()){
					c.resetAntCount();
				}
				aco.getWorld().initAnts();
				ants = (ArrayList<Ant>)aco.getWorld().getAnts();
				reset = false;
				}
			while(antsWorking > 0){
				for(Ant ant: ants){
					if(!aco.getRunning()){
						return null;
					}
					if(!ant.getFinished()){
						ant.move();	
						aco.reduceWorking();
					}else{
						//if an ansst is finished, decrease the counter
						antsWorking--;
					}
				}
					aco.getWorld().decayPhero();
			}
			reset = true;
		}
	}
	
	
\end{lstlisting}
\caption{Snippet of code used to automatically execute the algorithm untill compeltion. $aco$ is the current instance of the $AntColonyOptimisation$ Class.}
\label{iterationThing}
\end{figure}

\begin{figure}[H]
\begin{lstlisting}
stepIndex = 0;
if(currentIter < iterations){
		running = true;
		Ant ant = world.getAnts().get(stepIndex);
		if(!ant.getFinished()){
			ant.step();
			this.notifyCanvas();
			if(ant.getUnvisted() == 0){
				ant.setFinished(true);
			}
			}else{
			reduceWorking();
			stepIndex++;
		}
			if(stepIndex >= world.getAnts().size()){
			//reset the ants and index
			world.initAnts();
			for(City c: world.getCities()){
				c.resetAntCount();
			}
			stepIndex = 0;
			agentsWorking = noOfAgents;
			currentIter++;
		}
	}
\end{lstlisting}
\caption{Snippet of code used to execute the algorithm on a step by step basis}
\label{stepM8}
\end{figure}

\begin{figure}[H]
\begin{lstlisting}
public static double linearInterpolateX(double x1,double x2, double mu)	{
	return(x1*(1-mu)+x2*mu);
}

public static double linearInterpolateY(double y1,double y2, double mu)	{
	return(y1*(1-mu)+y2*mu);
}
\end{lstlisting}
\caption{Snippet of code determine the cordinate values for both $x$ and $y$ for a given $\mu$ value.}
\label{linearCode}
\end{figure}

\begin{figure}[H]
\begin{lstlisting}
City start = null;
City destination = null;
	for(Ant ant: agents){
		if(!ant.getFinished()){
			for(City c: cities){
				if(ant.getCurrentIndex() == c.getIndex()){
					if(antImage != null){
						g2.drawImage(antImage, (c.getX() * 20) - 15, (c.getY() * 20) - 9, null);
					}else{
						g2.setColor(Color.GREEN);
						g2.fillOval((c.getX() * 20) - 5, (c.getY() * 20) - 5, 10, 10);
					}
				}
				if(ant.getMovementTracker()[0] != ant.getMovementTracker()[1]){
					if(c.getIndex() == ant.getMovementTracker()[0]){
						start = c;
					}
					if(c.getIndex() == ant.getMovementTracker()[1]){
						destination = c;
					}
					if(start != null && destination != null){
						g2.setColor(Color.CYAN);
						for(int i = 1; i < 5; i++){
							//cast to a double to stop integer behaviours rounding down to 0
							double result = ((double)i)/5;
							//draw an ellipse2D every 1/5 the way along the line
							float y = (float) (Globals.linearInterpolateY(start.getY(), destination.getY(), result));
							float x = (float) (Globals.linearInterpolateY(start.getX(), destination.getX(), result));
							Ellipse2D movement = new Ellipse2D.Float((x * 20.0f) - 5, (y * 20.0f) - 5, 10, 10);
							g2.fill(movement);
							g2.draw(movement);
					}
				}
			}
		}
	}
	//reset the values
	start = null;
	destination = null;
\end{lstlisting}
\caption{Snippet of code used to visualise an Ants movement. $g2$ is an instance of $Graphics2D$.}
\label{antWalk}
\end{figure}

\begin{figure}[H]
\begin{lstlisting}
bestRoute = model.getWorld().getBestRoute();
	if(bestRoute != null){
		for(int i = 0; i < bestRoute.size(); i++){
			for(City c: cities){
				if(bestRoute.get(i) == c.getIndex()){
					if(i + 1 < bestRoute.size()){
						for(City c2: cities){
							if(bestRoute.get(i + 1) == c2.getIndex()){
								g2.setColor(Color.RED);
								//make the best line slightly thicker than the rest so it stands out more
								g2.setStroke(new BasicStroke(2));
								g2.drawLine(c.getX() * 20, c.getY()*20, c2.getX() * 20, c2.getY() * 20);
							}
						}
					}
				}
			}
		}
	}
\end{lstlisting}
\caption{Snippet of code used to display the best route to the user.}
\label{bestCode}
\end{figure}

\begin{figure}[H]
\begin{lstlisting}
for(EliteAntData data: eliteAnts){
	System.out.println(data.getEliteRoute());
	best = data.getEliteRoute();
	double e = (1/4) * numberOfCities;
	for(int i = 0; i < best.size(); i++){
		if(i + 1 < best.size()){
			pheromone[best.get(i)][best.get(i+1)].addToNewPhero
			(pheromone[best.get(i)][best.get(i+1)].getNewPhero() + e);
		}
	}
}
\end{lstlisting}
\caption{Snippet of code used to depsoit pheromone along the elite routes. An $EliteAntData$ Object is used to represent an Elite route}
\label{elitePhero}
\end{figure}

\begin{figure}[H]
\begin{lstlisting}
Random r = new Random();
while(numberOfUphill > 0){
	int index = r.nextInt(cities.size());
	City temp = cities.get(index);
	index = r.nextInt(cities.size());
	if(!(temp.getUphilRoutes().contains(index)) && (index != temp.getIndex())){
		temp.addToUphil(index);
		numberOfUphill--;
		distanceMatrix[temp.getIndex()][index] = (distanceMatrix[temp.getIndex()][index] * 2);
		invertedMatrix[temp.getIndex()][index] = 1/distanceMatrix[temp.getIndex()][index];
	}
}
\end{lstlisting}
\caption{Snippet of code used to initialise uphill paths.}
\label{suchUphill}
\end{figure}

